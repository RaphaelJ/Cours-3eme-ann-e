\documentclass[a4paper,12pt,french]{article}
\usepackage[utf8]{inputenc}
\usepackage[francais]{babel}
\usepackage[babel=true]{csquotes}
\usepackage{url}
\usepackage{listings}

\pagestyle{headings}

\title{InpresZon}
\author{Raphaël Javaux}
\date{}

\begin{document}

\maketitle 

\begin{abstract}
   Dossier d'aide à l'installation des différents composants de la plate-forme
d'e-commerce \enquote{InpresZon}.
\end{abstract}

\tableofcontents

\newpage

\section{Présentation des différents composants}

    \paragraph{}
    La plate-forme d'e-commerce \enquote{InpresZon} utilise divers logiciels
pour ses différents composants :

    \paragraph{Site web}
    Les deux instances du site web (une américaine et une belge) sont conçues à
l'aide du framework web Django\footnote{\url{https://www.djangoproject.com/}},
utilisant le langage de programmation
Python\footnote{\url{http://www.python.org/}}. 

    \paragraph{Bases de données}
    Les quatre bases de données (une américaine, une belge, une anglaise ainsi
qu'un dépôt de Data Warehouse) utilisent le SGBDR Oracle 11g Release
2\footnote{\url{
http://www.oracle.com/us/products/database/enterprise-edition/}}. La
communication entre les bases de données et les sites web se fait à l'aide du
pilote cxOracle\footnote{\url{http://cx-oracle.sourceforge.net/}}.

    \paragraph{Importation CouchDB}
    Un programme console permet l'importation de produits stockés dans une base
de données CouchDB\footnote{\url{http://couchdb.apache.org/}}. Ce programme,
écrit en Haskell\footnote{\url{http://www.haskell.org/}}, est compilable à
l'aide de GHC\footnote{\url{http://www.haskell.org/ghc/}}.

    \paragraph{Importation fnac.com}
    Un second programme console permet l'importation d'articles (musiques, films
ou livres) vendus sur le site \url{http://www.fnac.com}. Ce dernier est écrit en
F\#\footnote{\url{http://research.microsoft.com/fsharp}} et extrait les
informations des produits sous forme de documents XML. Un script Bash utilisant
le programme ncftp\footnote{\url{http://www.ncftp.com/}} a été créé pour
transmettre ces documents aux différents dépôts XDB des bases de données.

    \paragraph{Nouveaux stocks}
    L'envoi des fichiers XML représentant les nouveaux arrivages en stock se
fait à l'aide d'un script Bash utilisant, à nouveau, ncftp.

    \paragraph{Data Warehouse}
    Un dépôt de Data Warehouse, utilisé à des fin de data mining, est alimenté
à l'aide :
    \begin{itemize}
        \item de deux procédures stockées PL/SQL et de DB/LINK homogènes pour
les bases de données belge et anglaise ;
        \item d'un programme écrit avec le langage de programmation
Clojure\footnote{\url{http://www.clojure.org/}} pour la base de données
américaine, générant des fichiers XML pouvant être transmis au dépôt XDB de la
base de données de Data Warehouse à l'aide d'un script Bash.
    \end{itemize}

    \paragraph{}
    Les différentes opérations de data mining peuvent être effectuées à l'aide
d'un programme console Scala\footnote{\url{http://www.scala-lang.org/}}.

\section{Sites web}
    \paragraph{Python}
    Il est nécessaire d'installer la version 2 de l'interpréteur
Python\footnote{Django et le pilote Oracle ne sont pas encore compatibles avec
Python 3.}. Celui-ci est déjà installé sur la plupart des distributions
GNU/Linux (il se trouve dans les dépôts le cas échéant) ainsi que sur Mac OS X.
L'installeur pour Microsoft Windows est disponible sur
\url{http://www.python.org/}\footnote{L'ensemble de l'installation n'a été
testée que sous GNU/Linux. Il est fort probable que quelques démarches
supplémentaires soient nécessaires sur d'autres systèmes.}.
    
    \paragraph{Django}
    Le framework Django est disponible dans les dépôts des distributions
GNU/Linux. Si ce n'est pas le cas, on peut l'obtenir sur le site officiel :
\url{https://www.djangoproject.com/}.

    \paragraph{Serveur web}
    Un serveur web est fourni avec Django. Pour lancer le serveur web,
il est nécessaire de se positionner dans le répertoire du site web à servir
(\textit{be/} ou \textit{usa/}) et d'exécuter la commande suivante :
    \begin{lstlisting}
        $ python2 manage.py runserver 8080
    \end{lstlisting}
    Ceci va démarrer un serveur web sur le port 8080.
    
    \paragraph{}
    Il est également possible d'utiliser un site web créé avec Django sur le
serveur web Apache à l'aide du module
WSGI\footnote{\url{http://code.google.com/p/modwsgi/}}.

\section{Bases de données}
    \paragraph{Installation}
    La base de données Oracle 11g Release 2 est disponible en téléchargement
sur le site
d'Oracle\footnote{\url{
http://www.oracle.com/technetwork/database/enterprise-edition/downloads/index.ht
ml}} pour une utilisation non commerciale. 

    \paragraph{Pilote cxOracle}
    Il est nécessaire d'installer le pilote cxOracle pour faire communiquer les
sites web avec les bases de données. Le pilote est téléchargeable à l'adresse
\url{http://cx-oracle.sourceforge.net/}. Si la base de données Oracle n'est pas
installée sur la même machine que celle hébergeant les sites web, il sera
nécessaire d'installer l'InstantClient
d'Oracle\footnote{\url{
http://www.oracle.com/technetwork/database/features/instant-client/}}.

    \paragraph{Utilisateurs}
    Il est nécessaire de créer quatre utilisateurs pour les quatre bases de
données :
\textit{be}, \textit{usa}, \textit{uk} ainsi que \textit{marketing}. Par
défaut, les différents composants sont configurés pour utiliser le mot de passe
\textit{pass} pour l'ensemble de ces comptes. Celui-ci est modifiable dans les
fichiers \textit{settings.py} des répertoires \textit{be}, \textit{usa},
\textit{uk} et \textit{marketing}.
    Les utilisateurs \textit{be}, \textit{usa} et \textit{marketing} doivent
posséder les droits sur un dépôt XDB\footnote{Voir documentation sur
l'installation et la configuration d'un dépôt XDB.}.

    \paragraph{Tables}
    Les tables sont créées à partir de classes Python à l'aide de Django. Cette
manière de procéder a plusieurs avantages :
   \begin{itemize}
        \item la description des tables est indépendante du logiciel de gestion
de base de données utilisé (Django est officiellement compatible avec Oracle,
SQLite, PostgreSQL et MySQL) ;
        \item Django est capable de générer des formulaires HTML à partir de
ces classes ainsi que les vérifications associées (email valide, longueur des
chaînes \dots), ce qui évite certaines tâches répétitives ainsi que les erreurs
de programmation qui y sont associées ;
        \item Django fournit un langage pour écrire des requêtes (utilisant
des objets Python) indépendant du SGBD utilisé (fonctionnalité non employée
pour InpresZon).
    \end{itemize}

    \paragraph{}
    Pour créer les tables, il suffit de se positionner dans le répertoire de la
base de données (\textit{be/}, \textit{usa/}, \textit{uk/} ou
\textit{marketing/}) et d'exécuter la commande suivante :
    \begin{lstlisting}
        $ python2 manage.py syncdb
    \end{lstlisting}

    \paragraph{Packages}
    Le répertoire \textit{packages/} contient les scripts SQL utilisés pour
créer les différents packages des quatre bases de données. Le script
\textit{exceptions.sql} doit être exécuté sur les bases \textit{be} et
\textit{usa} et contient le module de gestion des exceptions.

\section{Importation CouchDB}

    \paragraph{Haskell}
    Le programme d'importation doit être compilé avec un compilateur Haskell.
GHC est le compilateur le plus utilisé et est disponible sur la plupart des
systèmes d'exploitation. Un paquetage contenant GHC ainsi que divers outils est
mis à disposition sous le nom d'\enquote{Haskell Platform} à cette adresse :
\url{http://hackage.haskell.org/platform/}. La plupart du temps, ce dernier est
directement disponible dans les dépôts des distributions GNU/Linux.

    \paragraph{ODBC}
    Le programme Haskell d'importation des articles utilise ODBC pour accéder
aux bases de données Oracle. Sous GNU/Linux, c'est le programme unixODBC
(disponible sur la plupart des distributions dans les dépôts) qui fournit la
connexion ODBC.

    \paragraph{}
    Il est indispensable de configurer un driver nommé \enquote{oracle}, car
c'est celui-ci qui sera utilisé par le programme d'importation, dans le fichier
\textit{/etc/odbcinst.ini} :
    \begin{lstlisting}
        [oracle]
        Description     = Oracle ODBC driver for Oracle 11g
        Driver          = /opt/instantclient/libsqora.so.11.1
        Setup           =
        FileUsage       =
        CPTimeout       =
        CPReuse         = 
    \end{lstlisting}
    La variable \textit{Driver} doit pointer vers le pilote \textit{libsqora.so}
de l'InstantClient ou de l'installation d'Oracle.

    \paragraph{}
    Haskell Platform n'est pas fourni avec les bibliothèques permettant la
communication avec les drivers ODBC et la base CouchDB,  cependant on peut
utiliser la commande \textit{cabal} (le gestionnaire des libraires de Haskell)
pour installer celles-ci :
    \begin{lstlisting}
        $ cabal update
        $ cabal install HDBC-odbc
        $ cabal install CouchDB
    \end{lstlisting}
    Cabal va se charger de récupérer les sources des libraires, de les compiler
et de les installer.

    \paragraph{Utilisation}
    Le programme d'importation CouchDB se nomme \textit{alimentation.hs} et se
compile avec GHC :
    \begin{lstlisting}
        $ ghc -O2 alimentation.hs
    \end{lstlisting}
    Ceci va créer un exécutable nommé \textit{alimentation}.

    \paragraph{}
    Il est possible de changer les paramètres de connexion aux serveurs
Oracle et CouchDB dans les premières lignes du ficher
\textit{alimentation.hs}\footnote{Toute modification exigera une re-compilation
du fichier \textit{alimentation.hs}.}.
    
\section{Importation fnac.com}

    \paragraph{F\#}
    Le programme d'importation est écrit en F\#. Le compilateur F\# est
disponible sur le site de Microsoft :
\url{http://research.microsoft.com/fsharp}. Celui-ci est déjà inclus avec
Visual Studio 2010. Sous GNU/Linux, il faudra en plus installer
Mono\footnote{\url{http://www.mono-project.com/}} (disponible dans la plupart
des dépôts) pour exécuter des binaires .NET. Microsoft supporte F\# sous Windows
et GNU/Linux.

    \paragraph{ncftp}
    ncftp est un programme console permettant de se connecter à un serveur FTP
depuis le terminal. Il est installable depuis les dépôts de la plupart des
distributions GNU/Linux.

    \paragraph{Utilisation}
    Le programme d'importation se nomme \textit{alimentation\_indep.fs} et se
compile avec cette commande (ou via Visual Studio pour Windows) :
    \begin{lstlisting}
        $ fsharpc -O -r System.Xml.Linq.dll alimentation_indep.fs
    \end{lstlisting}
    Ceci va créer un exécutable console nommé \textit{alimentation\_indep.exe}
exécutable directement sous Windows ou avec la commande \textit{mono} sous
GNU/Linux.

    \paragraph{}
    L'envoi des fichiers XML sur le dépôt XDB se fait avec le script Bash
\textit{alimentation\_indep\_xml.sh}.

\section{Nouveaux stocks}
    
    L'arrivage de nouveaux stocks se fait à l'aide du script Bash utilisant
ncftp. Le script se nomme \textit{nouveaux\_stocks.sh} et s'utilise comme suit :
    \begin{lstlisting}
        $ ./nouveaux_stocks.sh fichier_arrivage.xml
    \end{lstlisting}

\section{Data Warehouse}

    \paragraph{Chargement des données depuis BE et UK}
    Le package \textit{ChargementDonnees} contient deux procédures,
\textit{ChargementBe} et \textit{ChargementUk} pour importer dans le Data
Warehouse les données des bases belge et anglaise, respectivement.

    \paragraph{Chargement des données depuis USA}
    
        \subparagraph{Clojure}
        Le chargement des données se trouvant dans USA se fait à l'aide
d'un programme écrit pour le langage de programmation Clojure. Clojure utilise
la JVM. Le compilateur Clojure ainsi que la JVM sont disponibles dans la plupart
des dépôts GNU/Linux. Il est de plus nécessaire d'installer et de configurer
JDBC pour Oracle\footnote{Voir syllabus sur JDBC.}.

        \subparagraph{Utilisation}
        Le programme de chargement se nomme \textit{chargement\_usa.clj}  
et s'exécute avec Clojure :
        \begin{lstlisting}
            $ clj chargement_usa.clj
        \end{lstlisting}   

    \paragraph{Consultation des résultats}

        \subparagraph{Scala}
        Le programme console écrit en Scala permet d'effectuer diverses
opérations de forage de données sur le Data Warehouse. Tout comme Clojure, Scala
utilise la JVM et JDBC et son compilateur est disponible sur la plupart des
dépôts GNU/Linux.

        \subparagraph{Utilisation}
        Le programme de forage se nomme \textit{forage.scala}
et s'exécute avec Scala :
        \begin{lstlisting}
            $ scala forage.scala
        \end{lstlisting}

\appendix

\section{Sauvegardes}
    
    \paragraph{}
    Etant donné la nature financière des transactions effectuées sur la
plate-forme d'e-commerce, il est nécessaire d'établir des méthodes de
sauvegardes robustes. 

    \paragraph{}
    Cependant, deux remarques viennent à l'esprit :
    \begin{itemize}
        \item les produits des bases de données américaine et anglaise sont
recopiés sur le dépôt belge à une cadence de vingt articles ;
        \item les commandes et les utilisateurs sont recopiés en temps réel
entre les bases belge et américaine.
    \end{itemize}

    \paragraph{}
    Ainsi, certaines données sont déjà dupliquées à deux endroits
différents\footnote{A la condition que les différentes bases soient installées
sur des machines distinctes.}. Ceci est particulièrement intéressant dans le
cas des commandes et des utilisateurs, car il s'agit clairement des données les
plus délicates du système, et elles sont répliquées en temps réel (c'est à
dire sans risque de perte pour ces tables lorsque seulement l'un de ces deux
dépôts rencontre un problème).

    \paragraph{}
    Néanmoins, ça n'est pas suffisant, car même s'il est très peu probable que
les deux dépôts rencontrent une panne simultanément, il serait assez aisé à un
pirate de s'attaquer à la seconde base s'il a réussi à s'introduire dans la
première (celles-ci sont quasi identiques et de plus liées par un DBLINK).

    \paragraph{}
    La première chose à faire semble être de configurer les bases de données en
mode ARCHIVELOG afin de permettre des sauvegardes à chaud\footnote{Il est
impensable de stopper la plate-forme durant plusieurs minutes lors des
sauvegardes.} incrémentales cohérentes. La sauvegarde à chaud devrait se faire
durant les heures creuses. Les tables liées aux utilisateurs et leurs commandes
devront être sauvegardées plus régulièrement, même si cela se fera au détriment
des performances. 

\end{document}