\documentclass[a4paper,12pt,french]{article}
\usepackage[utf8]{inputenc}
\usepackage[francais]{babel}
\usepackage[babel=true]{csquotes}
\usepackage{url}
\usepackage{listings}

\pagestyle{headings}

\title{Labo 4 : Web Service REST}
\author{Raphaël Javaux}
\date{}

\begin{document}

\maketitle 

\paragraph{Création du projet}
Dans NetBeans, créer un projet \textit{Application Web}.
Il est ensuite possible de rajouter des Web Services REST via \textit{Nouveau
fichier}. Dans le cadre du laboratoire, j'ai créé un Web Service REST depuis la
base de données existante.

\paragraph{Définition du Web Service}
Les différentes annotations \textit{@Path, @Produces, @Consumes, @POST,
@GET, @PUT et @DELETE} permettent de spécifier les attributs des opérations.
\textit{@PathParam} permet de récupérer les paramètres du chemin de l'URL
et \textit{@QueryParam} permet de récupérer les paramètres de l'URL
(\textit{?page=} par exemple). Les méthodes doivent être publiques.

\begin{lstlisting}
@Path("webservicerest.vol")
public class VolFacadeREST extends AbstractFacade<Vol> {
    public VolFacadeREST() {
        super(Vol.class);
    }

    @PUT
    @Path("{id}")
    public void edit(
        @PathParam("id") Integer id,
        @QueryParam("landed") Integer
landed
    ) {
        [...]
    }
}
\end{lstlisting}

\paragraph{Test du Web Service}
Tout comme pour les Web Services SOAP, Netbeans permet de directement
déployer et tester les Web Services sur un serveur d'application dans une
application JSP.

\end{document}