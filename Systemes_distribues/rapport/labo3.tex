\documentclass[a4paper,12pt,french]{article}
\usepackage[utf8]{inputenc}
\usepackage[francais]{babel}
\usepackage[babel=true]{csquotes}
\usepackage{url}
\usepackage{listings}

\pagestyle{headings}

\title{Labo 3 : Web Service SOAP}
\author{Raphaël Javaux}
\date{}

\begin{document}

\maketitle 

\paragraph{Création du projet}
Dans NetBeans, créer un projet \textit{Application Web}.
Il est ensuite possible de rajouter des Web Services SOAP via \textit{Nouveau
fichier}. Sélectionner création d'un nouveau Web Service et non basé sur un
fichier WSDL existant.

\paragraph{Définition du Web Service}
La classe qui contiendra le Web Service aura l'annotation
\textit{@WebService} et toutes ses méthodes d'opération devront porter
l'annotation \textit{@WebMethod}. Ces méthodes doivent être publiques.

\begin{lstlisting}
@WebService(serviceName = "WebServiceSOAP")
public class WebServiceSOAP {
    @WebMethod(operationName = "HistoriqueVols")
    public @WebResult(name="Vol") Vol[] HistoriqueVol() 
    {
        [...]
    }
}
\end{lstlisting}

\paragraph{Classes de retour}
Les classes utilisées comme valeur de retour pour les opérations du Web
Service devront comporter un constructeur par défaut ainsi qu'une annotation
\textit{@XmlElement} sur les accesseurs. 

\begin{lstlisting}
public class Vol {
    private Date _depart; 
    @XmlElement(name="depart")
    public Date getDepart() {
        return _depart;
    }

    public void setDepart(Date depart) {
        this._depart = depart;
    }
}
\end{lstlisting}

\paragraph{Handler}
La classe utilisée comme handler pour les messages SOAP doit dériver de la
classe générique \textit{SOAPHandler} (le paramètre générique de la classe
définit le type du handler).
Le handler doit ensuite être enregistré. Cela peut se faire à l'aide de
l'opération \textit{Configurer les handlers} du menu clic-droit du Web Service.

\begin{lstlisting}
public class AuthHandler
        implements SOAPHandler<SOAPMessageContext>
{
    public boolean handleMessage(SOAPMessageContext context)
    {
        [...]
        return true;
    }
}
\end{lstlisting}

\paragraph{Test du Web Service}
Netbeans permet de directement déployer et tester le Web Service sur un
serveur d'application (Glassfish, Tomcat \dots{}) dans une application JSP
(via un clic droit sur le Web Service). Les exceptions qui seraient apparues
lors d'exécution du Web Service se trouveront dans les logs du serveur
d'application.

\paragraph{Utilisation du Web Service en Ruby}
La bibliothèque \textit{Savon} peut être utilisée pour accéder à des Web
Services SOAP en Ruby. Il suffit de lui fournir l'URI du fichier WSDL. Les
exemples sur le site officiel sont suffisants pour utiliser l'API.

\begin{lstlisting}
require "savon"

client = Savon::Client.new do |wsdl, http, wsse|
    wsdl.document = wsdl_uri
    wsse.credentials user, pass
end

client.request(:wsdl, :historique_vols)
\end{lstlisting}

\end{document}