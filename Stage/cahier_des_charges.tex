\documentclass[a4paper,12pt]{article}
\usepackage[utf8]{inputenc}

\pagestyle{headings}

\title{Cahier des charges}
\author{Stage de fin d'études de Raphael Javaux}
\date{}

\begin{document}

\maketitle 

\paragraph{}
L'étudiant sera intégré dans une équipe de recherche travaillant sur un projet
de détection de la somnolence lors de la conduite automobile.

\paragraph{}
L'objectif du travail sera l'élaboration d'un algorithme de détection et de
suivi des caractéristiques de l'oeil humain (position de la paupière, de l'iris
et de la pupille) afin d'établir le niveau de somnolence de la
personne observée.
L'image fournie en entrée sera au préalable ajustée sur l'oeil, l'étudiant ne
devra pas gérer la détection et l'extraction de celui-ci. 

\paragraph{}
L'objectif final sera la mise en production de l'algorithme sur un matériel
embarqué pour une utilisation en temps-réel, il est donc nécessaire que la
méthode réalisée puisse fonctionner à plus de 200 images par seconde sur un
ordinateur personnel moyen-haut de gamme.

\paragraph{}
L'étudiant utilisera la librairie de vision par ordinateur OpenCV à l'aide du
langage de programmation C++. Il y aura donc une période d'apprentissage dédié
à l'utilisation de celle-ci.

\paragraph{}
Un des critères importants pour le choix final des algorithmes concerne les
performances de la détection en fonction de la qualité des images fournies en
entrée. L'étudiant devra expérimenter plusieurs techniques afin de sélectionner
la plus efficace (dans le cadre d'une utilisation en temps réel).

\end{document}