\documentclass[a4paper,12pt]{article}
\usepackage[latin1]{inputenc}

\pagestyle{headings}

\title{Data Warehouse}
\author{Raphael Javaux}
\date{December 2011}

\begin{document}

\maketitle

\begin{abstract}
   This article introduces data warehousing and shows the differences with
traditional database management. Data warehousing and data mining are the
techniques of extraction of correlations and facts in corporations data and
take part in the business intelligence process.
\end{abstract}

\newpage

\tableofcontents
\newpage

\section{Introduction}
    \paragraph{}
    As of today, more and more information can be stored and modern computing
power enable complicated statistics variables to be calculated and analysed.

    \paragraph{}
    Today's corporations maintain huge data about their activities. This
large bank of information is a real forgotten treasure for most companies.

    \paragraph{}
    Most of time, all the knowledge a firm needs for marketing and economical
strategy can be found in its operations history.

    \paragraph{}
    \textit{How to make this product more attractive ? What is the best
period to launch an advertisement campaign for a given product ?} These are
examples of questions for which the data warehouse field is able to respond.

    \paragraph{}
    The purpose of this paper is to introduce the reader to data warehouse.
There is no prerequisite, excepts basics in databases management.

\newpage

\section{Definition}
    \paragraph{}
    Since their early days, computers were build to manage and analyse large
amounts data.

    By automatising information handling, these machines were quickly proved
required for any knowledge driven company.

    \paragraph{}
    By years, firms have accumulated and kept digital materials for about
everything in touch with their business, even when data were no longer
required for every day operations.

    Quickly, computer scientists have seen the benefits of these ``old''
information for affair decisions and have been looking for new techniques for
storage, centralization and extraction of facts, with success. First data
warehouses were born.

    \paragraph{}
    Because of information scattering and the really different requirements of
facts extraction (which is called \textit{data mining}), databases in use for
statistics are not the same as \textit{operational databases} (databases in
use in business parts of the company). These databases are called \textit{data
warehouse} and are only used for data mining.

\newpage

\section{Comparing Operational Databases and Data Warehouse}
    \paragraph{}
    Data warehouses don't meet the same needs as operational databases.

    \paragraph{Conception}
    Both systems are not accessed the same way. Data warehouses users will
\textbf{ask for statistics about data} (using dimensions and
grains\footnote{\textbf{Dimensions} are the different facts available in a data
warehouse, for example \textit{sales} or \textit{users registrations}.
\textbf{Grains} are the levels of detail of a given dimension, for example
\textit{sales per week} or \textit{sales per day}}) while operational databases
users will \textbf{ask for data} (using relations between records).

    Thus, DBMS software in use for operational databases are not the same as
systems in use in data warehouse. Though Oracle have some extensions for
data-mining, MySQL doesn't. Software designed for data warehouse exist in
both proprietary and free software worlds. 

    \paragraph{Performance}
    Data warehouses don't have to give the same performance level. For example,
an operational database must be able to answer to hundreds of concurrent
requests while a data warehouse will only receive a few queries per day.

    \paragraph{Security}
    Whereas in a operational database, security is the top priority, this is
not so important in a data warehouse. A lot of critical operations related
to operational databases, like financial transactions, don't exist in a data
warehouse.

    \paragraph{Storage requirement}
    Because data warehouses need to store a large history of operations, they
expect more devices for storage.

    \paragraph{Real time}
    Data warehouses don't need to be immediately updated with new data. Most of
the time, a data warehouse is synchronised with an operational database a few
times per week.

\newpage

\section{Conclusion}
    \paragraph{}
    Whereas data warehouse systems store structured data, just like traditional
relational database systems, they are really different by design.

    \paragraph{}
    The way of working with decisional data is really different from the way of
working with operational data. Both approach require divergent infrastructures,
software and skills.

\end{document}